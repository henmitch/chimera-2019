Chimera states in brain models have often been linked loosely to
unihemispheric sleep,
seizures,
and other brain behaviors \cite{Abrams2008,Panaggio2015,Martens2013,Abrams2004,Shanahan2010,Hohlein2019,Bansal2019,Chouzouris2018}.
In most cases,
these connections are made in off-hand remarks to introduce the concept of a chimera state,
but serious connections between these phenomena are rarely drawn.
However,
there are some notable cases of investigations of chimera states in brains.

A example of a chimera state being investigated in neural models was an exploration of chimera states on a network of Hindmarsh-Rose neurons (\cref{eq:hr_x,eq:hr_y,eq:hr_z}) \cite{Santos2017}.
This model was simulated on the connectome of a cat.
Parameter space for the two connection strengths $\hra$ and $\hrb$ was explored.
Chimera states are most prevalent for low values of $\hrb$, the inter-cortex connection strength \cite{Santos2017}.
This is unsurprising.
If the inter-cortex connection strength is too high as compared to the intra-cortex connection strength, the coupling acts globally instead of nonlocally.
This means that each cortex has less holding it together than pulling it apart, allowing the system to descend into asynchrony.

Further, with increasing input current $I_{j}$ (and increasing noise in the input current), chimera states give way to incoherence.
This also intuitively makes sense.
As the input current increases, its significance relative to the coupling also increases.
Thus, the oscillators have no reason to synchronize.
And, of course, adding noise will simply amplify the effect \cite{Santos2017}.

%%% Local Variables:
%%% mode: latex
%%% TeX-master: "../../ms"
%%% End:
