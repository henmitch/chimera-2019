Chimera states have been observed in many other systems, whether they be purely mathematical, biological, electrical, or mechanical \cite{Shanahan2010,Abrams2004,Andrzejak2016,Hizanidis2016,Kuramoto2002,Martens2013,Panaggio2015,Santos2015,Santos2017,Kruk2018,Xie2014}.
One of the most common ways that chimera states are discussed is in regards to seizures.

As is often the case with emergent phenomena, it is wildly impractical to simulate the collective behavior of a brain by simulating its constituent neurons.
Since the human brain has approximately $10^{11}$ neurons with $10^{14}$ synapses, direct simulation is too computationally intensive.
To better understand the dynamics of large portions of the brain, many researchers have turned to the techniques of thermal and statistical physics \cite{Breakspear2017},
resulting in \textit{neural field models} and \textit{neural mass networks}.
The first treats the brain as a continuous sheet of cortex, within which activity obeys wave equations.
The second represents the brain as a discrete graph of cortices, or a network of coupled oscillators.
The network used for the coupling of the oscillators is determined by the brain's connectivity matrix, or connectome.
An example of a neural mass network model is the modified Hindmarsh-Rose model [\cref{eq:hr_x,eq:hr_y,eq:hr_z}], which we discuss later.

One of the benefits of a neural mass network model is that its outputs are similar to those of an electroencephalograph, or EEG.
The EEG is a device used to record the electrical activity of the brain.
Electrodes are placed in specific areas on the scalp, and then changes in voltage are measured from neural masses beneath the skull.
Much of the signal is distorted and attenuated by the bone and tissue between the brain and the electrodes, which act like resistors and capacitors.
This means that, while the membrane voltage of the neuron changes by millivolts, the EEG reads a signal in the microvolt scale \cite{Kandel2013}.
The EEG also has relatively low spatial and temporal resolution (16 electrodes for the whole brain, and a sampling rate of \SI{33}{\ms}).
However, when properly treated, neural mass models make for effective predictors of the output from EEGs \cite{Taylor2012,Leistritz2007}.
This is useful, as EEGs are the main tool used to detect and categorize seizures.

\subsubsection{Seizure \AE tiology}
\label{sec:intro_seizures_aetiology}
Researchers define seizures as abnormal, excessive, or overly-synchronized neural activity \cite{Kandel2013,Baier2012}.
It is important to distinguish between seizures and epilepsy, as the two are often conflated.
Seizures are an acute event, whereas epilepsy is a chronic condition of repeated seizures.
While classification schemes vary, all center around the division between generalized and focal seizures.

The focus of the present work is focal seizures, which start in one part of the brain (the seizure focus).
They are typically preceded by auras such as a sense of fear, or hearing music, and often manifest as clonic movement of the extremities.
In many cases, they secondarily generalize, spreading to the entire brain.
This can make focal seizures and primary generalized seizures hard to distinguish, as a focal seizure can generalize rapidly after a brief aura.
This can lead to misdiagnoses and improper treatments.

%%% Local Variables:
%%% mode: latex
%%% TeX-master: "../../ms"
%%% End:
