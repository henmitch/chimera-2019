We coded the modified Hindmarsh-Rose model using Python (Python version 3.7.0, NumPy version 1.15.2, Pandas version 0.23.4, SciPy version 1.1.0), and integrated using a 4th-order Runge-Kutta with variable step size\footnote{Step size was determined by SciPy's internal algorithms, but was limited to a maximum of 0.01.} $\dd{t} < 0.01$.
We verified the code by reproducing the results of \cite{Santos2017}.
We ran the model for a time period of $T_{\text{sim}} = \bqty{-1000, 5000}$, where only times $T = \bqty{0, 4000}$ were saved.
We threw away the times $\bqty{-1000, 0}$ to eliminate transients.
The chimeras were extremely unlikely to be eliminated on such a time scale, due to the size of the network \cite{Wolfrum2011}.
We calculated the times $\bqty{4000, 5000}$ to facilitate analysis of the phase.

We computed the phase of the $j$th neuron in the resulting waveform as
\begin{equation}
  \label{eq:hr_phase}
  \phase_{j}(t)
  =
  2 \pi \times \frac{t - t_{i}}{t_{i + 1} - t_{i}},
\end{equation}
where $t_{i}$ is the time at which the $j$th neuron fires ($x_{j}$ crosses 0 in a positive direction) for the $i$th time\footnote{This is a similar measure for the phase as was used in \cite{Santos2017}, but allows for easier discrimination between physical and aphysical parameter sets.
  It is modified to keep $\phase_{j} \in \pqty{0, 2\pi}$ and to eliminate ambiguity about the meanings of the subscripts.
}.
In order for this calculation to be possible for all values in $T$, it was necessary to have each neuron fire at least once after $T$ had finished (i.e., there has to be some $t_{i + 1} \notin T$ in order to calculate the phase for times $t_{i} \leq t \leq t_{\text{max}} = 4000$).
The calculated time range went so far beyond $t_{\text{max}}$ so that any extremely slow-firing neurons were allowed to do so, to ensure that as much of parameter space was in the physical region (\cref{sec:results_aphysical}).
We then used phase to find the chimera and metastability indices of the result using \cref{eq:chimera} and \cref{eq:metastability} respectively.

We repeated this process for various parameter sweeps of $\hra \times \hrb$, summarized in \cref{tab:parameter_sweeps}.
Note that the step in each strength $i \in \Bqty{\hra, \hrb}$ is $\Delta i = \frac{i_{\text{max}}}{n_{i} - 1}$, due to the fact that the ranges are inclusive of both endpoints.
\begin{table}[ht]
  \centering
  \begin{tabular}{c | c | c}
    $\hra_{\text{max}}$, $\hrb_{\text{max}}$ & $\Delta \hra$ ($n_{\hra}$), $\Delta \hrb$ ($n_{\hrb}$) & Figure \\ \hline
    1.6, 0.4 & 0.0203 (80), 0.0211 (20) & --- \\
    3.2, 0.8 & 0.0405 (80), 0.0205 (40) & --- \\
    0.2, 0.1 & 0.00253 (80), 0.00256 (40) & \cref{fig:zoom} \\
    0.9, 0.9 & 0.0101 (80), 0.0101 (80) & --- \\
    1.0, 1.0 & 0.0101 (100), 0.0101 (100) & \cref{fig:aphysical_chimera}
  \end{tabular}
  \caption[Parameter sweeps]{The sweeps we used in evaluating the effects of $\hra$ and $\hrb$ on the chimera and metastability indices.
    All parameter sweeps started at $(\hra, \hrb) = (0, 0)$.
    We performed the $\bqty{0, 1} \times \bqty{0, 1}$ sweep 10 times, and averaged the resulting chimera-like indices.
  }
  \label{tab:parameter_sweeps}
\end{table}

Initial conditions were drawn from uniform distributions of $x_{j} \in \bqty{-2, 2}$, $y_{j} \in \bqty{0, 0.2}$, $z_{j} \in \bqty{0, 0.2}$.
We performed all simulations on the \href{https://www.uvm.edu/vacc}{Vermont Advanced Computing Core}, and is \href{https://github.com/henmitch/chimera-2019}{available online}\footnote{\href{https://github.com/henmitch/chimera-2019}{https://github.com/henmitch/chimera-2019}}.

%%% Local Variables:
%%% mode: latex
%%% TeX-master: "../../ms"
%%% End:
