Many mathematical models use the concept of coupled oscillators.
For example, circuits are often modeled as pendula or spring-mass systems.
Having large numbers of oscillators tied together often leads to strange behavior, following the adage ``more is different.''
An example of odd results observed in networks of coupled oscillators is a so-called ``chimera state,'' which is the coexistence of coherently synchronized oscillators and incoherent oscillators.

A concrete example of a chimera state was discovered using metronomes on swinging platforms \autocite{Martens2013}.
Researchers placed 15 metronomes each on two swinging platforms.
They then attached these swinging platforms together with springs.
When the spring attaching the platforms were of the right strength, they saw that all of the metronomes on one of the platforms were swinging together, while the metronomes on the other platforms were oscillating asynchronously.
This was the first observed chimera state in a network of mechanical oscillators.

It has been found that chimera states will only persist for infinite time with infinite oscillators.
In the real-world cases of finite oscillators, they are guaranteed to collapse into a state of complete synchrony \autocite{Wolfrum2011}.
Researchers often draw analogies between these collapses of chimera states and epileptic seizures \autocite{Andrzejak2016}.
Chimera states have often been observed in coupled oscillator models implemented on networks modeled after the connections in brains, but these models are often disconnected from reality \autocite{Santos2017,Lytton2008}.

My plan is to implement one of these coupled oscillator models on the human brain network, and see if the chimera state collapses produced match with data from epileptic seizures.

%%% Local Variables:
%%% mode: latex
%%% TeX-engine: xetex
%%% TeX-master: "../main"
%%% End:
