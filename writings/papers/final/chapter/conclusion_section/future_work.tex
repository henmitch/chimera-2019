The logical next step is to compare the simulated EEG traces to actual data collected from mice, with the aid of an epileptologist.
With further work in this direction, our research could potentially go from a mathematical curiosity to an applicable therapeutic and diagnostic tool.

Finding an instructive phase-space embedding of the Hindmarsh-Rose network would be challenging (seeing as it is a 639-dimensional system),
but would likely reveal potentially useful insights into the nature of the mechanisms underlying these systems.
The same could be said for Lyapunov analysis, as well as finding an informative way to create a bifurcation diagram and perform more in-depth bifurcation analysis.

Another way our work could be extended is by looking at chimera state collapse and its relationship to secondary seizure generalization.
However, it would be extremely computationally expensive, given the size of the system, and would therefore require some clever handiwork \cite{Wolfrum2011}.

Future work could also naturally be performed on better, more up-to-date connectomes \cite{Knox2019}.

%%% Local Variables:
%%% mode: latex
%%% TeX-master: "../../ms"
%%% End:
