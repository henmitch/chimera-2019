My process will follow three main steps.
The first is to computationally implement the HR model (\cref{eq:HRx,eq:HRy,eq:HRz}) on a functional human connectivity network (connectome), similarly to what was done on the cat connectome.
\begin{align}
  \label{eq:HRx}
  \dot{x}_{j} &= y_{j} - x_{j}^{3} + b x_{j}^{2} + I_{j} - z_{j}
                -
                \frac{\alpha}{n'_{j}} \sum_{k = 1}^{N}{G'_{j, k} \Theta(x_{k})}
                -
                \frac{\beta}{n''_{j}} \sum_{k = 1}^{N}{G''_{j, k} \Theta(x_{k})} \\
  \label{eq:HRy}
  \dot{y}_{j} &= 1 - 5 x_{j}^{2} - y_{j} \\
  \label{eq:HRz}
  \dot{z}_{j} &= \mu \bqty{s \pqty{x_{j} - x_{\text{rest}}} - z_{j}}
\end{align}
As part of the testing of this step, in fact, I plan to recreate the results of Santos \etal on the same cat connectome which they used for their work.
This development is unlikely to take very long, as it is a relatively simple task to write Python code to integrate the requisite differential equations on a network.
Additionally, there are several readily available human functional connectomes such as those from the Human Connectome Project and the Budapest Reference Connectome.

Once I have implemented this model for the human connectome, I will use parameter estimation techniques to find the circumstances that lead to chimera state evolution and collapse \autocite{Ramsay2007}.
This will allow me to generate various chimera states in the model, so as to later compare my simulations with real data.
This portion of the project is likely to take longer than the model implementation, as it will require running the HR model with parameter-matching code on top of it.

Finally, I will match the simulations to seizure data obtained from the Department of Neurology at the University of Vermont Medical Center.
This will involve filling out paperwork for the Internal Review Board, and is likely to take the longest of the three steps.
This will also require parameter matching techniques, and both quantitative and qualitative comparison between measured data and my simulations.
If the simulated patterns from the chimera states are similar enough to the data collected from actual seizures, then I will have shown the validity of the HR model under certain conditions.
Given the amount and breadth of the seizure data that the University has collected, if I am able to match these data, I will have shown in a general and universal sense which parameter values (both in the model and in the brain) lead to seizures.
The main difficulty in doing so is that I will need a way to determine that a fit is ``good enough,'' which is a determination that I will need to make from the literature and experts' opinions.

%%% Local Variables:
%%% mode: latex
%%% TeX-engine: xetex
%%% TeX-master: "../main"
%%% End:
