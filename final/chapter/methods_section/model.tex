The model used for this work was the modified Hindmarsh-Rose neural model\footnote{The modification is to add in the coupling, turning it into a network model instead of a single neuron.} \cite{Santos2017}.
\begin{align}
  \label{eq:hr_x}
  \dot{\hrx}_{j}
  &=
    \hry_{j}
    -
    \hrx^{3}
    +
    b \hrx_{j}^{2}
    +
    I_{j}
    -
    \hrz_{j}
    -
    \frac{\hra}{n'_{j}} \sum_{k = 1}^{N} G'_{j k} \Theta_{j}(\hrx_{k})
    -
    \frac{\hrb}{n''_{j}} \sum_{k = 1}^{N} G''_{j k} \Theta_{j}(\hrx_{k}) \\
  \label{eq:hr_y}
  \dot{\hry}_{j}
  &=
    1
    -
    5 \hrx_{j}^{2}
    -
    \hry_{j} \\
  \label{eq:hr_z}
  \dot{\hrz}_{j}
  &=
    \mu \pqty{s \pqty{\hrx_{j} - \hrx_{\text{rest}}} - \hrz_{j}}
\end{align}
where
\begin{equation}
  \label{eq:hr_sigmoid}
  \Theta_{j}(\hrx_{k})
  =
  \frac{\hrx_{j} - \hrx_{\text{rev}}}{1 + e^{-\lambda \pqty{\hrx_{k} - \theta}}}
\end{equation}
is the sigmoidal activation function, making it a neural mass model.
\Cref{tab:hr_params} shows the values and meanings of the symbols in the model.

\begin{table}[ht]
  \centering
  \begin{tabular}{c | c | l}
    Symbol & Value & Meaning \\ \hline
    $\hrx_{j}$ & --- & Membrane potential of the $j$th neural mass \\
    $\hry_{j}$ & --- & Associated with the fast processes \\
    $\hrz_{j}$ & --- & Associated with slow processes \\ \hline
    $b$ & 3.2 & Tunes the spiking frequency \\
    $I_{j}$ & 4.4 & External input current \\
    $\hrx_{\text{rev}}$ & 2 & Ambient reverse potential \\
    $\lambda$ & 10 & Sigmoidal activation function parameter \\
    $\theta$ & -0.25 & Sigmoidal activation function parameter \\
    $\mu$ & 0.01 & Time scale for variation of $z$ \\
    $s$ & 4 & Governs adaptation \\
    $\hrx_{\text{rest}}$ & -1.6 & Resting/equilibrium potential \\ \hline
    $\hra$ & Varied & Connection strength within cortices \\
    $n_{j}'$ & See \cref{fig:n_prime} & Number of connections within a cortex from the $j$th neuron \\
    $G_{j k}'$ & See \cref{fig:g_prime} & Intra-cortical connection matrix \\
    $\hrb$ & Varied & Connection strength between conrtices \\
    $n_{j}''$ & See \cref{fig:n_prime} & Number of connections between cortices from the $j$th neuron \\
    $G_{j k}''$ & See \cref{fig:g_prime} & Inter-cortical connection matrix
  \end{tabular}
  \caption[Hindmarsh-Rose Parameters]{The list of parameters used in modeling the Hindmarsh-Rose network.}
  \label{tab:hr_params}
\end{table}

This model was chosen due to the semanticity of its parameters, as well as its proven ability to exhibit chimera-like behavior as a neural mass model \cite{Santos2017}.
Additionally, the Hindmarsh-Rose model was not designed to emulate seizures, which provides further evidence for the theory that chimeras are a universal aspect of brain activity, as discussed in \cref{sec:lit_review_chimera_square_torus}.

%%% Local Variables:
%%% mode: latex
%%% TeX-master: "../../main"
%%% End:
