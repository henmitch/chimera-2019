The science and mathematics of synchronization are among history's most well-studied areas of research.
One of the earliest well-documented appearances of synchrony in unexpected places was observed in 1665 by Dutch physicist Christiaan Huygens, the inventor of the pendulum clock\todo{Put this in bibtex}.
He noticed that two clocks hung from the same beam would eventually synchronize with each other.
He supposed that this was due to minuscule energy transfers between the two clocks through the wooden beam.
This hypothesis was proven nearly 350 years later which shows that even the simplest-seeming synchronization behavior results from complex dynamics \cite{PenaRamirez2016}.

This behavior extends to larger systems that two clocks.
A classic demonstration in many classes on the mathematics of synchronization depicts the same phenomenon with more oscillators \cite{Pantaleone2002}.
One places a platform on top of a set of rollers, and places at least two metronomes on that platform (see \cref{fig:metronome_demo} for a drawing).
When these metronomes are started with the same frequency, out of phase with each other, over time their phases drift until they synchronize.
\begin{figure}[ht]
  \centering
  \begin{tikzpicture}
    \draw (0, 0) circle (0.5);
    \draw (4, 0) circle (0.5);
    \draw[pattern=north east lines] (-2, 0.5) -- (6, 0.5) -- (6, 1) -- (-2, 1) -- (-2, 0.5);
    \draw (-4, -0.5) -- (8, -0.5);
    \draw (-0.5, 1) -- ++(0.2, 2) -- ++(0.6, 0) -- ++(0.2, -2);
    \draw (0, 2.5) circle (0.05);
    \draw[fill] (0, 1.5) circle (0.1);
    \draw (0, 2.7) -- (0, 1.5);

    \draw (1.5, 1) -- ++(0.2, 2) -- ++(0.6, 0) -- ++(0.2, -2);
    \draw (2, 2.5) circle (0.05);
    \draw[fill] (2, 1.5) circle (0.1);
    \draw (2, 2.7) -- (2, 1.5);

    \draw (3.5, 1) -- ++(0.2, 2) -- ++(0.6, 0) -- ++(0.2, -2);
    \draw (4, 2.5) circle (0.05);
    \draw[fill] (4, 1.5) circle (0.1);
    \draw (4, 2.7) -- (4, 1.5);

  \end{tikzpicture}
  \caption[Synchronization demonstration]{The classic demonstration of Huygens synchronization.  When the metronomes are set running, they eventualy synchronize due to the light coupling provided by the platform's ability to roll.}
  \label{fig:metronome_demo}
\end{figure}

One example of more complex behavior arising from similar mechanisms is the coexistence of synchrony and asynchrony within a system of identical coupled oscillators, a phenomenon known as a chimera state \cite{Kuramoto2002,Abrams2004}.
The existence of these chimera states are surprising, as they represent asymmetry within symmetric systems.
The first time this behavior was observed was in a ring of nonlocally coupled oscillators \cite{Kuramoto2002}.
The model is expressible in one dimension as
\begin{equation}
  \label{eq:kuramoto}
  \pdv{t} A(x, t)
  =
  \pqty{1 + i \omega_{0}} A
  -
  \pqty{1 + i b} \abs{A}^{2} A
  +
  K \pqty{1 + i a} \pqty{Z(x, t) - A(x, t)}
\end{equation}
where
\begin{equation}
  \label{eq:kuramoto_coupling}
  Z(x, t)
  =
  \int{G(x - x') A(x', t) \dd{x'}}
  \qand
  G(y)
  =
  \frac{\kappa}{2} e^{-k \abs{y}},
\end{equation}
which reduces to the phase equation
\begin{equation}
  \label{eq:kuramoto_phase}
  \pdv{t} \phi(x, t)
  =
  \omega
  -
  \int{G(x - x') \sin(\phi(x, t) - \phi(x', t) + \alpha) \dd{x'}}
  \qq{where}
  \tan(\alpha)
  =
  \frac{b - a}{1 + a b}.
\end{equation}
When numerically simulated, this system quickly falls into a chimera state (\cref{fig:kuramoto_chimera}).
\todo{Fill in time and position.}
\begin{figure}[ht]
  \centering
  \begin{subfigure}{0.6\textwidth}
    \caption{The time series of the Kuramoto simulation.}
    \label{fig:kuramoto_overhead}
  \end{subfigure} %
  \begin{subfigure}{0.3\textwidth}
    \caption{A snapshot at $t = $.}
    \label{fig:kuramoto_snapshot}
  \end{subfigure}
  \caption[Kuramoto simulation]{The results of a simulation of a Kuramoto oscillator, as described in \cref{eq:kuramoto_phase}.
    A 4th-order Runge-Kutta solver ($\dd{t} = 0.01$, $t_{\text{max}} = 1000$) was run on a system of 512 oscillators.
    \Cref{fig:kuramoto_overhead} shows the entire time series of the simulation.
    The behavior represented there is quite complex, with several distinct qualitative changes to the patterns in the system.
    However, in-depth analysis of this system is beyond the purview of this work.
    \Cref{fig:kuramoto_snapshot} shows a snapshot of the state of the system at $t = $.
    Note the juxtaposition of asynchronous (oscillators RANGE) and synchronous (oscillators RANGE).
  }
  \label{fig:kuramoto_chimera}
\end{figure}

Since then, chimera states have been found in simpler systems still.
One of the simplest is the Abrams model, two populations of identical oscillators with a stronger coupling strength within the populations than between them \cite{Abrams2008}.
The equation describing this system is given as
\begin{equation}
  \label{eq:abrams}
  \dv{\theta_{i}^{\sigma}}{t}
  =
  \omega
  +
  \sum_{\sigma' = 1}^{2} \frac{K_{\sigma \sigma'}}{N_{\sigma'}} \sum_{j = 1}^{N_{\sigma'}} \sin(\theta_{j}^{\sigma'} - \theta_{i}^{\sigma} - \alpha)
  \qq{where}
  K
  =
  \bmqty{\mu & \nu \\ \nu & \mu}
  \qand
  \sigma \in \Bqty{1, 2}.
\end{equation}
In this model, $\mu$ represents the intra-population strength, and $\nu$ represents the inter-population strength, meaning $\mu > \nu$.
Time can be scaled such that $\mu + \nu = 1$.
If $\mu - \nu$ is not too large, and $\alpha$ is not too much less than $\frac{\pi}{2}$, then this system can produce chimera states.
\todo{Insert pictures}

A similar system was also analyzed in the physical world \cite{Martens2013}.
Two swinging platforms were coupled together with springs of variable spring constant $\kappa$, and 15 metronomes---all tuned to the same frequency---were placed on each platform.
For a wide range of values of $\kappa$, all of the metronomes on one platform would synchronize, while the metronomes on other platform would remain asynchronous.
This system is directly analogous to the Abrams model.
The metronomes on the same platform are coupled through the motion of the swing, which heavily influences the motion of the metronomes.
This intra-community coupling is represented by $\mu$ in the Abrams model.
The metronomes on opposite platforms are coupled through the springs, which is a much weaker interaction, represented in the Abrams model by $\nu$.

Chimera states have been observed in many other systems, whether they be purely mathematical, biological, electrical, or mechanical \cite{Shanahan2010,Abrams2004,Andrzejak2016,Hizanidis2016,Kuramoto2002,Martens2013,Panaggio2015,Santos2015,Santos2017,Kruk2018,Xie2014}.
\todo{Find a transition.}

%%% Local Variables:
%%% mode: latex
%%% TeX-master: "../../main"
%%% End:
