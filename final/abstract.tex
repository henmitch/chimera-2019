Chimera states---the coexistence of synchrony and asynchrony in a nonlocally-coupled network of identical oscillators---are often sought as a model for epileptic seizures.
This work investigates that connection, seeking chimera states in a network of modified Hindmarsh-Rose neurons connected in the graph of the mesoscale mouse connectome.
After an overview of chimera states for neurologists,
and an overview of neurology for mathematicians,
previous connections between chimera states and seizures are reviewed in the current scientific literature.
The model was found to be of sufficient quality to produce superficially epileptiform activity.
The limitations of the model were investigated, depending on the strength of connections between subcortices within a cortex and between cortices.
A wide swath of parameter space revealed persistent chimera states.
